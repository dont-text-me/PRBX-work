When reviewing the literature for this project, the available information fell into
two categories: on one hand, it was important to consult papers describing past attempts
at classifying coffee beans, especially those that proposed methods for doing it
at scale.
On the other hand, looking at the wider world of image classification
revealed useful details and approaches to classifier architecture that informed
the decisions made in this project.
Overall, considering the history of image
classification as a field and its applications in the coffee industry built the
information background that enabled the development of the prototypes in this
project.

The overall picture gathered from reviewing the literature related to the task at hand
suggests that identifying items as small as coffee beans is definitely
possible, although not without some degree of effort invested in the process.
Several
of the reviewed papers claimed great results, though for many, the data gathering
and cleaning process involved rare or expensive equipment.
The following sections will outline some of the most frequently used image classification algorithms
and their application to coffee bean quality control.

\section{Commonly used image classification algorithms}
\label{sec:lit-review-general}
\subsection{K-nearest-neighbours classifiers}
On the conceptually simpler side of image classification algorithms lies the
KNN classifier~\cite{knnOverview}, a method of unsupervised learning, where the
classifier ''learns`` from the differences between the datapoints rather than
their labels.
A KNN classifier's main benefit is the lack of training time:
instead, the images are compared to the others as soon as they are available.
This allows for rapid prototyping and a quick implementation time at the cost of
the resources needed to make each decision when running the model.
Despite this tradeoff,
versions of the KNN architecture remain a popular choice for image
classification to this day, and, with some modification, solve the large resource
requirements by utilising advanced data structures, such as in the case of the
KD-KNN algorithm~\cite{kdtreeKNN}.

Since the algorithm requires the calculation of ''distance`` between any given
datapoints, several metrics of calculation exist, such as Euclidean, Manhatan and
more.
The choice of the distance metric as well as the value of K must be picked
experimentally, with a technique such as grid searching.
Overall, the KNN algorithm
provides a quick and efficient way of prototyping an image classifier, however
the simplicity comes at a cost of the need to manually pick the necessary
hyperparameters to get the best out of the architecture.

An application of this classification method to a coffee-related problem can be seen in a 2019 paper by Oliveri et al.~\cite{hyperspectralGreenOliveri}.
Using a KNN classifier to assign the closest category to a given bean image, the authors
aimed to classify green (unroasted) coffee beans.
Apart from the ''normal`` class, three further defects were identified, including dehydrated beans, beans extracted from
underdeveloped cherries and ''black`` beans, which were extracted from cherries that prematurely fell off the tree and died.

The images were taken in batches, using a hyperspectral scanner, which captures light
beyond the visible frequency range.
The capturing
of images was automated using a conveyor belt feeding rows of beans of the same category
under the scanner.
While the extra equipment required for this approach
introduces extra costs and may be better fit for a commercial application, the technique
of splitting an image of a row of beans into groups of individual images, all
resized to the same dimensions provided a great way to simplify and speed up the
data gathering process for this project.

The authors did not develop any physical apparatus to separate the defective beans
in their paper, however they were able to process images of groups of beans, highlighting
defective ones on the original image.
This approach fits really well with the
constraints and aims of this project, as being able to identify defects and determine
their place on the original image would already lead to a significant saving of effort
and time for the potential users.

Another strength of Oliveri et al's paper is in their iterative and transparent approach
to the development of the model.
The authors were one of the few who mentioned
cross validation as part of their project and explained the choice of the number
of neighbours for their classifier.
Furthermore, they admitted that due to the
rarity of certain bean defects, they were not able to gather enough samples to add
them to the classifier as well as having to remove a number of beans from the sample
pool to make the ratio of beans in each class as even as possible.

While the above shortcoming, coupled with a lack of reporting on the species or
processing of the beans, identifies areas for improvement, the overall approach
shows great potential in identifying defects.
The application of the model and its
design have influenced the research process for this project, which, with a more
varied dataset, aims to explore the topic of KNN classification further.

Despite KNN classifiers' popularity, they are not without their drawbacks, with the main one relating to
the difficulty of representing image data in a less high-dimensional and space-ineffective way.
To convey enough information about a given image, complex pre-processing procedures, such as PCA or texture analysis may
be needed.

\subsection{Neural networks}
A more modern and complex solution can be found in neural networks.
These classifiers
attempt to mimic the human brain, where each neuron activates or ''fires`` only
if some condition (the ''activation function``) passes a certain threshold.

Neural networks can be used for more than image classification and are an
actively developing topic to this day, however some network architectures utilizing convolution layers are known
for their performance with images and are commonly used as the baseline for more
domain specific applications.
Some classic examples of such architectures include LeNet~\cite{leNetOverview}
and AlexNet~\cite{alexNetOverview}, which are often used for datasets containing
smaller (in terms of pixel resolution) images, making them a good potential
starting point for use in this project.
AlexNet in particular has shown great potential
with image datasets, such as CIFAR-10~\cite{cifar10}, which, given that the
number of roasted coffee defects roughly matches the number of classes in the
CIFAR-10 dataset, suggests that an architecture similar to AlexNet may be a good
candidate for this project.

A notable use case of neural networks for coffee bean classification was done by Chen et al.~\cite{hyperspectralChen},
who have achieved an accuracy of 98.6\% when
classifying green coffee beans.
While the purpose of their paper is slightly different,
focusing on optimising a different stage in the coffee supply chain, their approach
to data collection and processing provided valuable insights serving as a great starting
point for the implementation of this project.
Similar to this project, the authors have identified several green coffee
defects to classify: insect damage, black beans (ones that have prematurely
fallen off the tree and could not develop) and bean fragments.

Similarly to Olivieri et. al.'s paper, image gathering was automated and involved capturing
several light spectra to extract more information from each image compared to visible light photography.

The authors, using neural networks to classify the images, have provided several
classifier architectures.
While the highest-performing network in their study (a
3D-CNN) required leveraging the hyperspectral data provided by the scanner, the
2D-CNN architecture they described used only the spatial data of the images.
Interestingly,
the CNN-based classifiers boasted fast prediction time, with the authors developing
a real-time sorting device.
While physical prototypes are beyond the scope of
this project, knowing that near real-time classification speeds are possible with
this architecture suggests that this approach could fit the task at hand well.
Furthermore, Chen et al.\ employed a dimensionality reduction algorithm (PCA)
for their model, which has improved the overall accuracy when reducing the hyperspectral
data down to 3 components.
This increase in performance could also be
investigated and leveraged in this project.

Despite the high performance and large dataset in Chen et al.'s paper, the real-world
performance of any model could be improved by a more diverse dataset, showing either
more defects, more coffee varieties, or both.
This project aims to cover this
gap by utilising beans processed by various methods, from several distinct origins
and species.

A 2017 paper by Nasution and Andayani~\cite{manyRoastLevelsNasution} shows a
departure from the previous two paper both in its approach to data gathering as
well as the overall aim.
Instead of focusing on green coffee defects, they
instead focused on coffee roast levels, aiming to classify the beans' degree of
roast on a scale of 16 levels, from completely green to burnt.
The images were
gathered using a common smartphone camera and resized to the same dimensions, with
each image containing a single bean.
For their classifier, the authors chose a backpropagation
neural network, with the images first processed by a gray level co-occurrence
matrix (GLCM), a technique that allows to develop insights into textures of surfaces
from an image.

Overall, the authors claim an accuracy of 97.5\%, however the report lacks
transparency in the total number of samples as well as the way classes were
determined, with some roast levels only having a slight difference in hue.
With
such a fine-grained scale of roast levels, it is difficult to see how the authors
were able to find a sufficient quantity of beans at each level and classify them
without samples of one class mixing into others.
Despite this lack of
transparency, the paper still provides a valuable look at how a task similar to the
one in this project can be done without requiring complex and expensive equipment.
The results achieved by the paper, while being able to benefit from more clarity
in reporting, suggest the possibility of developing a classifier that:
\begin{itemize}
	\item Is able to work on visible-light images only

	\item Is able to pick up on small changes in the surface of the beans

	\item Differentiates roasted beans, which may have less colour variation than
	green beans with a similar degree of accuracy
\end{itemize}

A paper with one of the closest aims to the project at hand was written by Shao et al.\ in
2022\cite{rgbDeepLearningShao}.
Similar to this project, the paper aims to
classify roasted beans, with the authors identifying a similar number of bean defects.
Interestingly, the authors identify ''peaberry`` beans or beans with an exceptionally
small, round shape, as a defect,
whereas many coffee consumers and producers believe that such beans provide a
flavour advantage over regular beans and market them as the more desirable
species.
It should be noted that this paper highlights the usefulness of automated
bean classification: with a sufficiently well-trained model, a physical prototype
could be implemented to sort the beans, allowing the producers to discard or
separate the bean classes they deem valuable or defective.
In that regard, Shao
et al's paper is an important showcase of the economic benefits brought along by
automation of the coffee quality control process.

The authors have also provided an automated method of data gathering, utilizing a
rotating wheel to briefly hold the beans in front of a camera.
Similar to
Nasution et al.\cite{manyRoastLevelsNasution}, the authors worked with RGB images
and did not employ any light spectra beyond the visible one.

The authors have used an existing, well-researched CNN architecture to develop
their model, with 11 total layers, achieving an accuracy as high as 96\% with the
lowest score of 88\% across the seven identified classes.
Similar to commercial solutions
described in section~\ref{sec:qc-state-of-the-art}, the authors have used an air
compressor to automatically remove defective beans, proposing a lower-cost, more
customizable solution of coffee quality control.

While the authors did not make any mention of the number of beans of each
category they gathered or the breakdown of origins and processing methods of the
beans, the high total number (1700) of the beans suggest that there was sufficient
data to train the model.
A potential improvement could be seen in a more varied
dataset with a more even breakdown of classes.

Apart from the described usage of expensive and complex machinery to
automatically gather the images and remove the defective beans, the approach taken
by Shao et al.\cite{rgbDeepLearningShao} showcases a low-cost approach, requiring only a visible-light camera
and sufficiently powerful hardware to train the model.

While gathering data for a prototype project is a relatively straightforward task,
a real-life coffee roaster may have access to a much greater number of defects
and may be in search of a more resilient solution that requires less manual
interaction compared to manually laying the beans out to take images of them.

Unlike KNN classifiers, neural networks require significantly larger datasets as
well as training time to fit the model to the data at hand.
In exchange for the increased
training time, neural networks boast a much faster classification time, with some
even allowing real-time classification.
While not explicitly a goal of this
project, the potentially faster classification time would make a neural network-based
prototype a better fit for use in industry, allowing the classifier to be used
at scale.

Data normalization is an important point to consider when using image
classifiers.
For best performance, the images must be of similar dimensions and
free of background noise or obstructions.
Therefore, when gathering data for
this project, a significant amount of effort must be put into ensuring that each
bean is well-lit, on a clear background and that other beans do not protrude
into the image.

A possible
solution to this problem can be seen in a 2024 paper by Eron et al.
\cite{eronCoffeeCherryOnTrees}, proposing the use of image classifiers at two stages
of the process.
Their study takes a departure from the ones described in this
section, with the aim being the detection of defective coffee cherries while they
are still attached to the tree, before any processing has been done to them.

In order to get a good picture of each coffee cherry, the authors have utilised a
regular, visible-light camera to capture branches of the coffee plants in their
natural state i.e.\ without moving or placing them to get a better angle.
The
authors have compared several variations of the YOLO architecture to extract the
images of coffee cherries from the resulting photographs, picking the YOLOv7 architecture
in the end.

The resulting images were resized to the same dimensions and processed again,
this time with a KNN-based classifier.
The authors have identified four categories
of coffee cherries, with the classifier averaging at a 3.78\% error rate across the
classes.

While this paper is only partially relevant to the task at hand, its usage of a
KNN classifier further reinforces the theory of its suitability for this project,
as well as showcases a potential application of neural networks to assist in data
gathering if the prototype developed here is applied at scale.

\subsection{Transfer learning}
\label{sec:transfer-learning-litreview}
As larger datasets, such as ImageNet~\cite{imageNet}, containing over a million images in a thousand classes,
became available, it was theorised that training a model on large amounts of non domain-specific data would allow it
to gain ''knowledge`` that can then be applied on a more specific dataset.
A paper by Razavian et.\ al.\ suggests the use of large, pre-trained networks as ''feature extractors``, which,
combined with an additional fully-connected layer, can display great performance on domain-specific data with little training
required, reducing the resources and time needed to adjust a model to a particular task~\cite{transferLearning}.

\section{Summary}
\label{sec:lit-review-summary}
Overall, the papers mentioned above provide a
clear indication that classifying small items such as coffee beans is possible
using the state-of-the-art in image classification models.

KNN and Convolutional neural networks stand out as the best potential choices, with
their performance verified in several papers, each describing a slightly
different use case and a different dataset makeup.
Furthermore, data
augmentation seems to be a popular technique, ensuring that the classifier is able
to handle real-world data, where the images of beans are not guaranteed to be in
the same orientation.
Finally, transfer learning shows great potential for reducing the training time while extracting more
features from the images, which would increase the likelihood of fulfilling criterion~\ref{itm:goal1} in section~\ref{sec:research-aims}.

Some of the papers lacked clarity in their reporting of the makeup of their
datasets, with some only gathering coffee of one origin or variety as well as omitting
lesser-known coffee processing techniques or species.
This project aims to
develop a varied and diverse dataset, ensuring that any classifier trained on the
data is able to handle unconventional looking coffee beans as well as the well-known
varieties, making it a good fit for the ever-changing landscape of the coffee
industry.