As stated in the above section, the results gathered during this project suggest its overall success.
With MobileNet and ResNet architectures allowing near real-time classification speeds and with under 2 percent of normal beans
erroneously marked as defective, these networks are likely to dramatically reduce the workload of a coffee roaster should they employ such
a model in their operations.
Similarly, with both models classifying less than 2\% of quaker and insect/mould damaged beans as normal,
the model is highly unlikely to hurt the business/marketing potential of the product.

Despite these successes, this project remains a software-focused one, with practical, let alone commercial-scale solutions
extending beyond its scope.
Therefore, the main area of future work with this project would lay in developing a way to harness the power of the models
and apply it to a real-world operation.
Unlike with software projects, developing hardware is likely to require significant funds and expertise in different areas of knowledge,
especially when aiming to develop a low-cost, compact solution that is able to be distributed to even the smallest scale roaster.
Furthermore, the training dataset itself could be extended with more varied examples, allowing the models to gain even better
performance for each defect class.
A potential vision of the future of this project is a community-driven dataset, with each roaster uploading images of the defects
they come across in their work, allowing the entire roaster community to benefit from the shared knowledge.

Finally, it is also important to consider the fact that sorting roasted coffee happens incredibly late in the production chain.
As pointed out by Christopher Feran, a coffee production expert~\cite{ferranCoffeeEthics}, this means that relying exclusively on
this quality control method would both drive the up the costs of roasted coffee to café owners and home consumers without benefitting the
farmers, who tend to be paid the least out of all involved in the production.
With this in mind, it is important to think of the prototypes developed here as links in a chain, with quality control at
consistent, frequent points in the coffee production cycle leading to a more fair and sustainable industry, benefitting the
consumer at the same time.