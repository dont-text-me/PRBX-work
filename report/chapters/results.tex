\section{Loading and pre-processing the dataset}
\label{sec:loading-and-pre-processing-the-dataset}
As mentioned in section~\ref{subsec:data-pre-processing}, a PyTorch \verb|DataLoader| was used to label and load the images
for each classifier.
After the images were loaded, they were resized to the same $400 \times 400$ resolution, ensuring that most images are
enlarged rather than shrunk.
This was a conscious choice, as the textural features of some defects were quite small and harsh shrinking of the images
could have corrupted or outright removed that information.

After the images were set to a consistent size, the further processing differed slightly between the classifier types:
for the KNN classifier, no further processing was done.
This decision was based on the reasoning in section~\ref{subsec:knn-classifier} and aimed to preserve the colour values of the pixels.
For neural network-based classifiers, the processing consisted from a sequence of transforms from the Pytorch \verb|torchvision.transforms|
package and consisted of the following items:
\begin{itemize}
    \item Converting the images to the \verb|float32| data type to enable training on the M3 GPU cores
    \item For the training set, applying a random horizontal flip and a rotation to augment the dataset
    \item For the testing set, only applying a rotation to augment the test data due to its small size
\end{itemize}

After applying a stratified train/test split, with 80\% of images used for training, the class counts for the two sets are
shown in table~\ref{tab:finalTrainTestClassCuts}.
The split was applied with a constant random state value for repeatable results.
\begin{table}[h]
    \centering
    \begin{tabular}{lll}
        \toprule
        \textbf{Defect name} & \textbf{Training count} & \textbf{Testing count} \\
        \midrule
        Normal & 1048 & 263 \\
        Quaker & 782 & 196 \\
        Bean fragment & 237 & 59 \\
        Underroasted & 83 & 21 \\
        Burnt & 40 & 10 \\
        Insect/Mould & 38 & 9 \\
        \bottomrule
    \end{tabular}
    \caption{Finalised training and testing set class counts}
    \label{tab:finalTrainTestClassCuts}
\end{table}

As seen from the the above table, despite the initial aims of the project, the distribution of the classes was quite unbalanced.
While no ill effect on the KNN performance has been observed, such a skewed distribution is highly likely to negatively
affect the performance of a neural network due to the low chance of a smaller class' sample being picked in a given epoch.

While Pytorch provides facilities for sampling from a given list of indices and sampling with a manually set probability for each class,
there is no built-in method for combining the two, making it difficult to apply a weighted sampler to a subset of the total dataset.
The solution to this problem was found in a third-party library~\cite{imbalancedSampler} providing the \verb|ImbalancedDatasetSampler| class.
This sampler draws from a given list of indices with the probability of each class' sample being picked equal to
$P(Class_{sample} = C) = \frac{1}{\arrowvert\{x \in X \colon Class_x = C\}\arrowvert}$ for class $C$ and dataset $X$.

Overall, by combining the techniques of over and under sampling as well as data augmentation, the networks were able to learn on a relatively
evenly balanced and varied dataset.
\section{KNN classifier results}
\label{sec:knn-results}
The experiments with a KNN classifier followed the methods outlined in section~\ref{subsec:knn-classifier}.
The sets of hyperparameters used to tune the classifier were as follows:
\begin{itemize}
    \item Distance calculation metric $M = \{Euclidean, Manhattan, Canberra\}$
    \item Number of nearest neighbours $K = [1, 30]$
    \item Nearest neighbour calculation algorithm $A = \{KDTree, Brute-force\}$
    \item Number of bins in colour histogram $N = \{32, 64, 128, 256\}$
\end{itemize}

All but the last hyperparameter were added to a set of nested loops with the candidate classifier being initialized with the
chosen values.
The number of bins was set separately as it was applied to the dataset itself prior to fitting the classifier.

The experiments produced two noteworthy classifiers, whose hyperparameter choices and overall accuracy are presented in table~\ref{tab:knnResults}.
\begin{table}[h]
    \begin{tabular}{@{}llllll}
        \toprule
        & \textbf{\makecell{Distance\\metric}} & \textbf{K value} & \textbf{\makecell{Calculation\\algorithm}} & \textbf{\makecell{Histogram\\bin count}} & \textbf{\makecell{Overall\\accuracy(\%)}} \\
        \midrule
        \textbf{KNN-1} & Manhattan & 28 & KDTree & 32 & 75.6 \\
        \textbf{KNN-2} & Canberra & 7 & KDTree & 32 & 75.8 \\
        \bottomrule
    \end{tabular}
    \caption{Finalised KNN classifiers}
    \label{tab:knnResults}
\end{table}

As seen from the above table, both classifiers achieved a relatively high accuracy.
However, inspecting the per-class performance of these classifiers reveals a significant difference.
A metric that can be used to compare per-class performance is the f1-score:
$f_1 = \frac{2tp}{2tp + fp + fn}$, where $tp$, $fp$ and $fn$ are the true positive, false positive and false negative rates respectively.
It should be noted that even the f1 score may not show the full picture: it may be insightful to also look at the precision
($\frac{tp}{tp + fp}$) and recall ($\frac{tp}{tp + fn}$) values per each class.
These metrics for KNN-1 and KNN-2 are presented in table~\ref{tab:knnScores}.
\begin{table}
    \centering
    \begin{tabular}{*7l}
        \toprule
        \textbf{Class} & \multicolumn{3}{c}{KNN-1} & \multicolumn{3}{c}{KNN-2} \\
        \midrule
        {} & Precision & Recall & F1 & Precision & Recall & F1 \\
            \textbf{Normal} & 0.77 & 0.95 & 0.85 & 0.78 & 0.92 & 0.84  \\
            \textbf{Quaker} & 0.73 & 0.83 & 0.76 & 0.74 & 0.81 & 0.77 \\
            \textbf{Bean fragment} & 0.82 & 0.15 & 0.25 & 0.58 & 0.12 & 0.20 \\
            \textbf{Underroasted} & 1.00 & 0.05 & 0.09 & 0.71 & 0.57 & 0.63 \\
            \textbf{Burnt} & 0.00 & 0.00 & 0.00 & 1.00 & 0.30 & 0.46 \\
            \textbf{Insect/Mould} & 0.00 & 0.00 & 0.00 & 0.50 & 0.11 & 0.18  \\
        \bottomrule
    \end{tabular}
    \caption{KNN classifier evaluation metrics}
    \label{tab:knnScores}
\end{table}

The above data suggests that the canberra distance metric allows the classifier to gain better
performance with the under-represented classes, while slightly reducing the performance on larger ones.
This is an important insight, suggesting that the metric can be picked according to the classifier's use case:
the manhattan metric can provide a powerful sorter where one may only be concerned from separating defective and normal beans,
while the canberra metric can provide insights into the distribution of the defects themselves.
It is clear, however, that the colour histogram may not be a powerful enough technique to explain the differences between
the defects and that a more sophisticated solution may be needed if an understanding of defect frequencies (or, an even more accurate sorter)
is the goal.


\section{Compact CNN results}
\label{sec:compact-cnn-results}
Similar to the above section, several model architectures have been tested.
Unlike with KNN classifiers, the hyperparameter space for neural networks is much greater, with many options of loss functions,
optimizer algorithms and more.
This meant that a simple grid search would be incredibly time and resource intensive and that a more thoughtful approach was needed.

The first item considered was the optimizer selection, with the Adam~\cite{adamGrad} and Stochastic gradient descent (SGD) algorithms being
the main candidates due to their wide adoption in industry and strong results on existing datasets.
While both optimizers showed good results when training the models, the SGD optimizer was chosen in the end for its ability
to be customized with momentum (with a value of 0.9 used) and its more intuitive operation.

A loss function needed to be picked next - being a classification problem with multiple classes, binary methods were not applicable,
therefore the choice fell on the binary cross-entropy function, which at a high level, measures the extent to which the
probability distribution of the model's prediction matches that of the actual data.
It should be noted that the Pytorch implementation of this loss function differs slightly from the commonly seen definition,
requiring that the input to the loss function consists of the actual class labels rather than the probability of each class
for each item in a batch for best performance.

Finally, the batch size and epoch number were selected experimentally - initial attempts showed a significant boost in accuracy
(though, at the expense of longer training times) when a smaller batch size was used: switching from 32 down to 8 led
to the biggest increase, with little to no gain seen from further reductions.
The number of epochs was determined from the size of the batches and from observing the magnitude of the changes in the loss function
with the optimal number being 40 for the shuffleNet and mobileNet architectures.

The optimal learning rate for the optimizer was different depending on the number of the current epoch.
While larger ''jumps`` were acceptable and even welcome at the earlier stages, a large learning rate could ruin tens of epochs
of training nearing the end of the training loop.
Therefore, instead of attempting to pick a single learning rate value, a scheduler provided by PyTorch was used.
The chosen solution was an instance of the \verb|StepLR| class, halving the learning rate every 10 epochs for both
ShuffleNet and MobileNet.
Overall, the two selected architectures achieved a significant improvement in both overall accuracy and per-class
metrics over the KNN classifier, with the results presented in table~\ref{tab:cnn-small-scores}.
The total accuracy percentages were 84\% and 81\% for MobileNet and ShuffleNet respectively.

\begin{table}
    \centering
    \begin{tabular}{*7l}
        \toprule
        \textbf{Class} & \multicolumn{3}{c}{MobileNet V2} & \multicolumn{3}{c}{ShuffleNet} \\
        \midrule
        {} & Precision & Recall & F1 & Precision & Recall & F1 \\
        \textbf{Normal} & 0.93 & 0.95 & 0.94 & 0.92 & 0.94 & 0.93  \\
        \textbf{Quaker} & 0.77 & 0.90 & 0.83 & 0.82 & 0.75 & 0.78 \\
        \textbf{Bean fragment} & 0.93 & 0.50 & 0.64 & 0.77 & 0.63 & 0.69 \\
        \textbf{Underroasted} & 0.57 & 0.62 & 0.60 & 0.28 & 0.76 & 0.41 \\
        \textbf{Burnt} & 1.00 & 0.40 & 0.57 & 0.75 & 0.30 & 0.43 \\
        \textbf{Insect/Mould} & 0.50 & 0.11 & 0.20 & 0.00 & 0.00 & 0.00  \\
        \bottomrule
    \end{tabular}
    \caption{Compact CNN evaluation metrics}
    \label{tab:cnn-small-scores}
\end{table}

Overall, moving to a compact CNN has shown a great improvement in both accuracy and per-class metrics over the KNN
classifier.
Despite that, the performance in smaller classes is still somewhat lacking, with the underroasted, burnt and insect damaged
beans being frequently misclassified.
It is clear that the network is lacking information to build a knowledge of the features of these beans from the small
data samples the dataset provides.
The next section will look at addressing these shortcomings by applying transfer learning to a much larger model.
\section{Transfer learning results}
\label{sec:transfer-learning-results}
When selecting the models here, much less concern was given to the size of the model - the main criteria instead were
their performance on a known dataset like ImageNet~\cite{imageNet}.

The ResNet architecture stood out as having great performance on the dataset and a good track record of successful
transfer learning applications.
Several versions of the architecture exist, with 18, 34, 50 and 152 layers.
The first three were selected for experiments, with the extremely large size of the 152 layer version leading to excessively long
training loops and marginal gains over the smaller architectures in preliminary experiments.
For comparison, the MobileNet architecture from the previous section and Swin transformer~\cite{swinTransformer} were also trained,
with the latter serving as an example of a transformer network.

As mentioned in section~\ref{subsec:transfer-learning}, each network was adjusted to predict 6 classes and trained using a similar
iterative approach as described in section~\ref{sec:compact-cnn-results}.

While training the models, it was discovered that they benefited from a lower learning rate and more aggressive scheduling,
with the ResNet models performing best with the learning rate reduced by ten times every eight epochs.
The best performing optimizer and loss function were SGD and Cross entropy loss respectively, though, with slightly different settings
from the ones described in section~\ref{sec:compact-cnn-results}.

It should be noted that MobileNet and ResNet were trained entirely - that is, every weight was updated during training.
The Swin transformer on the other hand, had all its weights frozen (by marking them as not requiring gradient calculations in PyTorch)
except those in the final layer.
This was done as the extreme size of the Swin transformer network caused poorer performance on the smaller classes when all weights were updated -
retaining the weights trained on ImageNet allowed it to better classify the smaller bean categories.

\begin{table}[h]
    \centering
    \begin{tabular}{*{10}l}
        \toprule
        \textbf{Class} & \multicolumn{3}{c}{ResNet 18} & \multicolumn{3}{c}{ResNet 34} & \multicolumn{3}{c}{ResNet 50} \\
        \midrule
        {} & {Prec.} & {Rec.} & F1 & {Prec.} & {Rec.} & F1 & {Prec.} & {Rec.} & F1 \\
        \textbf{Normal} & 0.93 & 0.98 & 0.96 & 0.96 & 0.99 & 0.98 & 0.99 & 0.99 & 0.99\\
        \addlinespace[0.5em]
        \textbf{Quaker} & 0.93 & 0.89 & 0.91 & 0.94 & 0.92 & 0.93 & 0.95 & 0.92 & 0.94\\
        \addlinespace[0.5em]
        \textbf{\makecell[l]{Bean\\fragment}} & 0.90 & 0.90 & 0.90 & 0.88 & 0.83 & 0.85 & 0.94 & 0.86 & 0.90\\
        \addlinespace[0.5em]
        \textbf{\makecell[l]{Under\\roasted}} & 0.80 & 0.76 & 0.78 & 0.77 & 0.81 & 0.79 & 0.60 & 1.00 & 0.75 \\
        \addlinespace[0.5em]
        \textbf{Burnt} & 0.86 & 0.60 & 0.57 & 1.00 & 0.60 & 0.75 & 0.91 & 1.00 & 0.95\\
        \addlinespace[0.5em]
        \textbf{\makecell[l]{Insect/\\mould}} & 0.83 & 0.56 & 0.67 & 0.67 & 0.89 & 0.76 & 1.00 & 0.67 & 0.80 \\
        \bottomrule
    \end{tabular}
    \caption{ResNet architecture evaluation metrics}
    \label{tab:resnet-scores}
\end{table}

Overall, the ResNet and MobileNet architectures showed the best results, with the 34 layer version achieving the best accuracy, though at the cost of a slightly longer
training time.
The best results were achieved on the white-background dataset (as described in section~\ref{subsec:data-pre-processing}),
with the per-class accuracy metrics presented in table~\ref{tab:resnet-scores}.
The overall accuracy for the models was 92\%, 94\% and 95\% for the 18, 34 and 50 layer architectures respectively, with the
pre-trained MobileNet also achieving a 95\% result.

The same metrics for the pre-trained MobileNet and Swin transformer networks are displayed in table~\ref{tab:transfer-results-2}.
The overall accuracy scores achieved were 95\% and 85\% for EfficientNet and Swin transformer respectively.

\begin{table}[h]
    \centering
    \begin{tabular}{*7l}
        \toprule
        \textbf{Class} & \multicolumn{3}{c}{MobileNet (pre-trained)} & \multicolumn{3}{c}{Swin transformer} \\
        \midrule
        {} & Precision & Recall & F1 & Precision & Recall & F1 \\
        \textbf{Normal} & 0.97 & 0.98 & 0.98 & 0.88 & 0.95 & 0.91  \\
        \textbf{Quaker} & 0.95 & 0.97 & 0.96 & 0.79 & 0.87 & 0.83 \\
        \textbf{Bean fragment} & 0.91 & 0.85 & 0.88 & 0.86 & 0.71 & 0.78 \\
        \textbf{Underroasted} & 1.00 & 0.90 & 0.95 & 1.00 & 0.19 & 0.32 \\
        \textbf{Burnt} & 0.89 & 0.80 & 0.84 & 0.00 & 0.00 & 0.00 \\
        \textbf{Insect/Mould} & 1.00 & 0.67 & 0.80 & 0.75 & 0.33 & 0.46  \\
        \bottomrule
    \end{tabular}
    \caption{ResNet alternatives evaluation metrics}
    \label{tab:transfer-results-2}
\end{table}
With the exception of the Swin transformer, the application of transfer learning brought significant improvements in
under-represented class performance as well as further increases in overall accuracy.
It is important, however, to consider the real-world application potential of each of the classifiers.
The next section will take a critical look at the classifiers' performance in the context of a real-world system.
\section{Evaluation}
\label{sec:evaluation}
While the high accuracy of the classifiers developed during the project makes a relatively strong case for it being a success,
it is still important to take a deeper look at the weaknesses and imperfections that might affect their usefulness in the industry.

First, it is important to consider the impact each defect has on the finished product: while both coffee roasters whose
samples were used in the project agreed that a defect-free product is the ultimate goal, they mentioned placing much more
effort and attention into removing quaker and insect/mould damaged beans than other defects.
This is due to the fact that while under and over-roasted coffee beans do affect the flavour of the cup, they do not
do so to nearly the same extent.
In other words, a small number of beans with no defects other than a wrong roast level is unlikely to damage their company's
reputation or economic prospects.

Second, the success criteria for a classifier can differ depending on the user's own goals - while some may only look for
a separation between normal and defective classes, others may wish for more thorough statistics on their product, with
the latter audience paying more attention to the model's performance in every class.

In any case, a balance of precision, recall and accuracy for the ''Normal`` class are equally
important for a classifier's viability: while an imprecise model will fail to identify normal beans leading to
a waste of good product and a negative impact on the finances of a roaster, a model with poor recall will mark too many
defective beans as normal, affecting the quality of the resulting product and the roaster's marketing potential.

With this in mind, the three kinds of models developed here can be evaluated.

The KNN classifier's performance leaves a lot to be desired - while KNN-1 achieved a good recall score for the ''Normal``
class, its precision is relatively low: a batch of coffee processed using this classifier will likely be sorted too agressively
with many good beans needlessly discarded.
Furthermore, the performance on smaller classes is quite poor, with only the ''Quaker`` classes achieving an F1 score above
0.5.
Despite this, the KNN approach still has its strenghts, which mainly relate to its simplicity and transparency -
the general principle behind it is likely to be understood even by a non-technical audience and the dimensionality reduction
method is relatively straightforward and relies on simple observations of the beans.
Using a highly optimised algorithm such as the one implemented in \verb|SkLearn| can also dramatically reduce the resources
needed for a single prediction, even though the scaling potential of this classifier is slightly lower due to the need to
compare each incoming image to the entire dataset.

Switching to a simple neural network architecture results in a much more viable model: with the MobileNet architecture
being the best performer in that category, the precision and recall scores for the ''Normal`` class



