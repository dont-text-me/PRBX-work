The popularity of coffee has shown steady growth over the last few years,
with the coffee industry going through several ''waves`` of priority shifts.
The latest shift in consumer focus has been towards the clarity of flavour and the complex tasting notes that
can only be extracted through careful sourcing, processing and roasting.
While green coffee production is usually a medium-to-large scale operation, coffee roasters, particularly ones targeting a
''specialty`` market tend to be very small companies, often roasting a few kilograms at a time.
Due to this, automated quality control for roasted coffee is out of reach for a significant number of coffee roasters,
for reasons including prohibitive pricing, noise, space constraints and energy consumption, requiring roasters to commit
significant manual and mental labour into manually sorting each batch.

The Specialty Coffee Association (SCA) is the main authoritative body in the coffee industry, with strict requirements
for beans that may earn the ''specialty`` badge.
In order to market their coffee as specialty, the roasters must ensure a near complete lack of defective beans in their
product, with the lack of availability of QC equipment preventing many smaller roasters from targeting the specialty market with
their product, hurting both the consumers, the roasters and smaller-scale farmers, whose livelihood depends on their crop.

To propose a fix to this issue, this project identified several requirements for a small-scale image classifier,
with the ideal solution not requiring many computational resources to develop and operate, as well as exhibiting a high degree
of precision and recall as to avoid both waste of normal beans and erroneously including defective beans in the final product.

In a collaboration with two UK-based roasters, a dataset of normal and defective roasted beans has been developed, amassing
over 2700 beans belonging to nine species and processed with six different techniques.
An image processing pipeline has been implemented using a computer vision library, digitizing and pre-processing the images.

The resulting dataset has been used to train a suite of classifiers, ranging in complexity from a KNN-based classifier combined with a grid-search
approach to select the best hyperparameters, to a pre-trained state-of-the-art neural network,
with the MobileNet V2~\cite{mobileNet} architecture displaying the best results.

Fine-tuned on the developed dataset, the model shows potential for use in low-power or real-time systems as well as high accuracy,
achieving an overall score of 95\% over 6 total classes with
1.90\% of normal beans falsely classified as defective and 2.03\% of defective beans classified as normal, making the model
a powerful tool that can assist even the smallest-scale operations at a low computational cost.

Further work has been identified as the development of a hardware prototype as well as in further developing the dataset,
with a proposal of a community-driven shared knowledge base, with roasters working together to represent the variety of coffee
available on the market.
