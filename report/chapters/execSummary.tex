The popularity of coffee has shown steady growth over the last few years.
With the coffee industry going through several ''waves``, the latest shift in consumer focus has been towards
the clarity of flavour and the complex tasting notes that can only be extracted through careful sourcing, processing and roasting.
While green coffee production is usually a medium-to-large scale operation, coffee roasters, particularly ones targeting a
''specialty`` market tend to be very small companies, often roasting a few kilograms at a time.
Because of this, automated quality control for roasted coffee is out of reach for a significant number of coffee roasters,
for reasons including prohibitive pricing, noise, space constraints and energy consumption, requiring roasters to commit
significant manual and mental labour into manually sorting each batch.

In a collaboration with two UK-based roasters, a dataset of normal and defective roasted beans has been developed, amassing
over 2700 beans belonging to nine species and processed with six different techniques.
An image processing pipeline has been implemented using a computer vision library, digitizing and pre-processing the images.

The resulting dataset has been used to train a suite of classifiers, ranging in complexity from a KNN-based classifier to
a pre-trained state-of-the-art network, with the MobileNet V2~\cite{mobileNet} architecture displaying the best results.

The model showed potential for use in low-power or real-time systems, achieving an overall accuracy score of 95\% over 6 total classes with
1.14\% of normal beans falsely classified as defective and 2.03\% of defective beans classified as normal, making the model
a powerful tool that can assist even the smallest-scale operations at a low computational cost.

Further work has been identified as the development of a hardware prototype as well as in further developing the dataset,
with a proposal of a community-driven shared knowledge base, with roasters working together to represent the variety of coffee
available on the market.
