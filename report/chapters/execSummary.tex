The popularity of coffee has shown steady growth over the last few years, with
the latest shift in consumer focus being towards the clarity of flavour and the complex tasting notes that
can only be extracted through careful sourcing, processing and roasting.
While green coffee production is usually a medium-to-large scale operation, coffee roasters, particularly ones targeting a
''specialty`` market tend to be very small companies, often roasting a few kilograms at a time.
Due to this, automated quality control solutions are out of reach for a significant number of coffee roasters,
due to their prohibitive pricing, large dimensions, noise and energy consumption, requiring roasters to commit
significant manual and mental labour into manually sorting each batch.

The Specialty Coffee Association (SCA) is the main authoritative body in the coffee industry, with strict requirements
for beans that may earn the ''specialty`` badge.
In order to market their coffee as specialty, the roasters must ensure a near complete lack of defective beans in their
product, with many smaller roasters struggling to target the specialty market with
their product, hurting both the consumers, the roasters and smaller-scale farmers, whose livelihood depends on their crop.

To propose a fix to this issue, this project identified several requirements for a small-scale image classifier,
with the ideal solution not requiring many computational resources to develop and operate, as well as exhibiting a high degree
of precision and recall as to avoid both waste of normal beans and erroneously including defective beans in the final product.

In a collaboration with two UK-based roasters, a dataset of normal and defective roasted beans has been developed, amassing
over 2700 beans belonging to nine species and processed with six different techniques.
An image processing pipeline has been implemented using the OpenCV~\cite{opencvLibrary} computer vision library, digitizing and pre-processing the images.

The resulting dataset has been used to train a suite of classifiers, ranging in complexity from a KNN-based classifier combined with a grid-search
approach to select the best hyperparameters, to a pre-trained state-of-the-art neural network.

The most notable results for each type of classifier were produced by the following models:
\begin{itemize}
    \item A KNN-based classifier, using the Canberra distance metric and the KDTree~\cite{kdtreeKNN} comparison algorithm,
    utilising colour histograms as a means of dimensionality reduction, achieving a total accuracy of 79.7\%
    \item A MobileNet V2~\cite{mobileNet} architecture trained from scratch, achieving a total accuracy of 84\%
    \item Another instance of a MobileNet, pretrained on the ImageNet~\cite{imageNet} dataset, achieving a total accuracy of 95\%
    \item A 50-layer ResNet~\cite{resNet} model, pretrained on the same dataset, achieving a total accuracy of 95\%
\end{itemize}

The pretrained MobileNet model displayed the most potential in real-world applications, with its low classification time
of 10.4 milliseconds allowing it to serve as tool for near real-time classification in a high throughput environment.
With less than 1.2\% of normal beans classified as defective, the model is unlikely to lead to excessive waste, and its
1.95\% error rate for classifying underdeveloped, mould or insect damaged beans makes it a powerful tool for adhering to
the SCA guidelines.
The ResNet model displayed even better results, with a 0.76\% error rate for normal beans and a 1.46\% rate of beans displaying
the aforementioned defects, though at the cost of nearly double the prediction time, with the model being better suited
for studying batches of beans in environments where real-time performance is less important.

Further work has been identified as the development of a hardware prototype as well as in further developing the dataset.
In particular, a study of misclassified beans revealed a need for more powerful lighting and higher resolution images in order to
capture more information of the beans' surfaces as well as gathering more samples of beans with mould and insect damage to
balance the dataset.
The project also proposes a community-driven shared knowledge base, with roasters working together to represent the variety of coffee
available on the market.

While the results achieved in this project achieve its aims, it is important to consider everyone involved in the production chain:
the models developed in this project are designed to be used at the very end of the production chain, with excessive reliance on
them alone driving the cost of the finished product up without benefitting the people responsible for farming, shipping and processing the beans.
Therefore, the models here must be used in a system of frequent quality control checks to ensure the fairest compensation for
everyone involved and the lowest strain on the environment.
