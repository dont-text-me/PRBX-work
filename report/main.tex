\documentclass{./styles/UoYCSproject}
\addbibresource{main.bib}
\newcommand{\eaccentmark}{é}
\graphicspath{{./figures}}
\author{Ivan Barinskiy}
\title{An evaluation of various image classification methods for identifying roast defects in coffee beans.}
\supervisor{Adrian Bors}
\BSc

\dedication{To all students everywhere}

%\acknowledgements{
%  I would like to thank my goldfish for all the help it gave me
%  writing this document.
%
%  As usual, my boss was an inspiring source of sagacious advice.
%}

\begin{document}
\pagenumbering{roman}
\maketitle
\listoffigures
\listoftables
%\renewcommand*{\lstlistlistingname}{List of Listings}
%\lstlistoflistings

\begin{summary}
THIS IS A DRAFT TODO TODO todo.
\end{summary}

\chapter{Introduction}
\label{ch:introduction}
\section{Coffee in society}
\label{sec:coffee-in-society}
In recent years, coffee has gained more and more popularity, cementing itself as one of the most widely consumed beverages.
Along with a surge in caf{\eaccentmark} openings, preparing coffee at home is now an activity enjoyed by more and more people, with the average consumer being more discerning and critical than ever.

The development of the coffee industry is often described as a series of ``waves''.
The first ``wave'' made coffee an everyday staple, focusing on convenience and high availability.
During the second wave, the idea of ``specialty'' coffee first entered the consumer vocabulary, with creative recipes, flavours and textures being popularised by emerging caf{\eaccentmark}s and chains.
The third wave, starting around 1980 and lasting to the present day saw a significant shift in consumer values:
rather than adding additional flavours in their coffee drinks, the consumers instead started noticing the taste characteristics of the beans themselves,
with simpler, more delicately flavoured drinks being preferred.
It is also during this time that coffee processing and roasting were moved into focus,
with the average consumer being more and more likely to differentiate and prefer a certain flavour profile over others.

The popularity of brewing ``specialty'' coffee at home has also risen during this time.
A coffee enthusiast is now more likely to own a grinder and one or more brewers, preferring to purchase whole-bean,
usually single-origin coffee, with the expectation of the resulting drink matching the advertised flavour and aroma notes.
This shift of consumer patterns has placed a lot of importance on coffee roasters and farmers,
who are now held to a higher standard than ever before.

The following section will discuss the challenges faced by coffee roasters and producers
as well as describe the ``state of the art'' of quality control in coffee production.

\section{Coffee production challenges}
\label{sec:coffee-production-challenges}


\chapter{Literature review}
\label{ch:litreview}


\chapter{Methods}
\label{ch:methods}

\chapter{Results}
\label{ch:results}

\chapter{Conclusion}
\label{ch:conclusion}


\appendix
\chapter{Appendix}
\label{ch:appendix}

\printbibliography

\end{document}