\documentclass[12pt]{article}
\usepackage[left=2cm, right=2cm]{geometry}
\usepackage[scaled]{helvet}
\renewcommand\familydefault{\sfdefault}
\usepackage[T1]{fontenc}
\usepackage{lipsum}
\usepackage{fancyhdr}
\lhead{Reflection document}
\rhead{Ivan barinskiy}
\begin{document}
\thispagestyle{fancy}
\section{Project specification}
\label{sec:section-1}
The choices of learning outcomes for my project were based on real conversations I had with people in the coffee industry.
Developing a low-cost, customizable solution was therefore a direct response to the problems they identified.
Initially, I was slightly reluctant to pick a project that would require me to build my own dataset as that felt like a significant
undertaking without any promise of success.

Regarding the initial project specification I proposed, it is clear that i may have put too much confidence into KNN-based classifiers,
with them performing the worst out of all classifiers I developed during the project.
However, this has been an insightful experience that allowed me to see how a neural network can easily outperform hand-engineered features
in a real-world context.
In that sense, the comparative aspect of my project specification was a good choice, allowing me to try my hand at every part
of a ''big-data`` project, from gathering the coffee beans to picking and training a neural network architecture.
Trying my hand at reducing the dimensionality of an image dataset has also showed me the power of even relatively simple methods such as colour histograms.
While I am sure that doing similar projects in industry would involve more complex programming and concepts, this project is very likely
to make a good baseline for the needed knowledge, making me familiar with commonly used concepts such as dataloaders, organising a dataset,
training on a GPU and presenting my findings for an audience.

On a more general level, setting what in my mind were very ambitious goals, has helped me reduce the feelings of anxiety when starting on a large project
and using a task board to track the writing of each chapter and implementing parts of my solution have solidified my project management skills after being introduced to them
during my year in industry.

Overall, I feel I was able to judge my abilities relatively well, identifying the fact that a hardware solution would likely be beyond
the scope of my project.
Achieving the other goals i set for myself however, has given me a sense of empowerment to try my hand at that task in my own time.

Finally, it is important to remark on the sense of achievement and pride I felt when communicating the results of my project back to my
coffee industry acquaintances - the excitement they showed for my results has removed any anxiety I had around the project's success.
Seeing the positive change possible from my work has made me excited to develop my skills further and made for a great reminder of the power of computer science.

If i were to attempt the project again, my main changes would be around stricter (or, more well-defined) benchmarks my solution would need to achieve -
in an ideal case, I would like to see my classifier compete against a coffee professional on the same set of samples.
In that case, I would seek some financing from the university to make sure that any professionals whose help I seek are fairly compensated -
this time, i got all the defective samples for free but made sure to purchase the ''normal`` beans for their usual retail price to ensure no one is taken advantage of.

Overall, I would call my choice of project specification a success, placing it at an intersection of my hobbies and areas of software engineering I felt I could get more experience in.
\section{Self-reflection}
\label{sec:section-2}
    \subsection{Problem solving}
    \subsection{Communication}
    \subsection{Responsibility}
\end{document}