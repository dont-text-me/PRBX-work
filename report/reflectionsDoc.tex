\documentclass[12pt]{article}
\usepackage[left=2cm, right=2cm, bottom=3cm]{geometry}
\usepackage[scaled]{helvet}
\usepackage[parfill]{parskip}
\usepackage[T1]{fontenc}
\usepackage{fancyhdr}
\usepackage{lipsum}
\usepackage{color}
\renewcommand\familydefault{\sfdefault}
\lhead{Reflection document}
\rhead{Ivan Barinskiy}
\setlength{\headheight}{15pt} % hides warning
\pagestyle{fancy}
\begin{document}
    \section*{Project specification}
    The choices of learning outcomes for my project were based on real conversations I had with people in the coffee industry.
    Developing a low-cost, customizable solution was therefore a direct response to the problems they identified.
    Initially, I was slightly reluctant to pick a project that would require me to build my own dataset as that felt like a significant
    undertaking without any promise of success.

    Regarding the initial project specification I proposed, it is clear that i may have put too much confidence into KNN-based classifiers,
    with them performing the worst out of all classifiers I developed during the project.
    However, this has been an insightful experience that allowed me to see how a neural network can easily outperform hand-engineered features
    in a real-world context.
    In that sense, the comparative aspect of my project specification was a good choice, allowing me to try my hand at every part
    of a ``big-data'' project, from gathering the coffee beans to picking and training a neural network architecture.
    Trying my hand at reducing the dimensionality of an image dataset has also showed me the power of even relatively simple methods such as colour histograms.
    While I am sure that doing similar projects in industry would involve more complex programming and concepts, this project is very likely
    to make a good baseline for the needed knowledge, making me familiar with commonly used concepts such as dataloaders, organising a dataset,
    training on a GPU and presenting my findings for an audience.

    On a more general level, setting what in my mind were very ambitious goals, has helped me reduce the feelings of anxiety when starting on a large project
    and using a task board to track the writing of each chapter and implementing parts of my solution have solidified my project management skills after being introduced to them
    during my year in industry.

    Overall, I feel I was able to judge my abilities relatively well, identifying the fact that a hardware solution would likely be beyond
    the scope of my project.
    Achieving the other goals i set for myself however, has given me a sense of empowerment to try my hand at that task in my own time.

    Finally, it is important to remark on the sense of achievement and pride I felt when communicating the results of my project back to my
    coffee industry acquaintances - the excitement they showed for my results has removed any anxiety I had around the project's success.
    Seeing the positive change possible from my work has made me excited to develop my skills further and made for a great reminder of the power of computer science.

    If I were to attempt the project again, my main changes would be around stricter (or, more well-defined) benchmarks my solution would need to achieve -
    in an ideal case, I would like to see my classifier compete against a coffee professional on the same set of samples.
    In that case, I would seek some financing from the university to make sure that any professionals whose help I seek are fairly compensated -
    this time, i got all the defective samples for free but made sure to purchase the ''normal`` beans for their usual retail price to ensure no one is taken advantage of.

    Overall, I would call my choice of project specification a success, placing it at an intersection of my hobbies and areas of software engineering I felt I could get more experience in.
    \pagebreak
    \section*{Self-reflection}
    One of this project's biggest challenges has been in the volume of work that was required to build the dataset.
    In particular, I feel like I underestimated the amount of time it would take to digitize the images, going as far as
    suggesting I photograph each bean individually to my supervisor in one of the early meetings.
    By developing an image processing pipeline, i was able to process up to 25 beans at a time, greatly reducing the required work.
    By doing so, I was also able to learn about blob detection and image masking, which are commonly used techniques in image processing,
    giving me useful skills for any future projects involving image classification.

    Later in the project, I identified a flaw in my dimensionality reduction algorithm, where I lost some image information
    by not separating the colour channels.
    This occurred due to me misreading the library's documentation when prototyping.
    After visualising the processed images, I was able to increase my accuracy by nearly 5\% by applying the algorithm
    correctly after re-visiting the documentation page and adjusting my code.
    This experience was important for several reasons: first, it taught me the importance of making sure I fully understand
    my issue before attempting to fix it.
    Second, visualising the data has given me valuable experience with libraries such as Matplotlib, a tool widely used in
    the industry and beyond, with the project greatly improving my familiarity with the tool and the quality of material I produce with it.
    Finally, I improved my skills in navigating technical documentation and my understanding of the underlying algorithm, with the improved
    accuracy score being solid proof of that.

    % communication
    This project's success, perhaps more than any other projects I have done, relied on my communication skills.
    Having to approach several people, with some of whom I never talked before, definitely felt like a challenge, allowing me to
    develop my skills in ''pitching`` my ideas and aims to a non-technical audience, which will certainly be useful in almost any area
    of software engineering I could find myself in the future.

    Another useful skill was in learning to condense my writing for the report itself: many times i would find myself
    writing excessively verbose text, and learning to critically evaluate my writing and identify the most important parts
    allowed me to be more efficient with my work and to communicate deeper insights of my findings while sticking to the page
    constraints.
    As an effect of this, I was able to shorten my introduction by more than a full page, removing the unrelated information and
    providing a clearer link between the study aims and the real-world problem behind the project.

    % responsibility
    Overall, I would say that my project had a relatively late start due to a lack of organisational skills and responsibility on my part.
    If I were to undertake such a project again, I would make sure to start gathering samples as early on in the process as possible.
    Despite this setback, I feel like my responsibility has grown over the course of the project: I learned to set realistic but
    demanding deadlines for myself and, in most cases, delivered the work earlier than planned, with my self-study skills growing stronger
    the further into the semester I went.
    Overall, my development in self-study, project management and workload evaluation have definitely improved over the course of the year,
    giving me useful soft skills that are definitely going to be useful in further work.

\end{document}